\documentclass[12pt]{article}

\usepackage{graphicx}
\usepackage{amsthm}
\usepackage{listings}
\usepackage{xcolor}

\theoremstyle{definition}
\newtheorem{example}{Example}[section]

\title{Electronic Devices and Circuits}
\author{Jimmie J. Cathey}

\begin{document}
\maketitle

\section{Circuit Analysis}

\subsection{Spice Elements}
The passive and active circuit elements introduced in the previous section are all avaialable in SPIC modeling; however, the manner of node specification and the voltage and current sense or direction are clarified for each element by Figure~\ref{fig:fig_01_02}. THe universal ground node is assigned the number 0. Otherwise, the node numbers $n_1$ (positive node) and $n_2$ (negative node) are positive integers selected to uniquely define each node in the network. The assumed direction of positive current flow is from the node $n_1$ t node $n_2$. The four controlled sources---voltage-controlled voltage source (VCVS), current-controlled voltage source (CCVS), voltage-controlled current source (VCCS), and current-controlled current source (CCCS)--- hae the associated controlling element also shown with its nodes indicated by $cn_1$ (positive) and $cn_2$ (negative). Each element is described by an \textit{element specification statement} in the SPICE netlist code. Table~\ref{tab:tab_01_01} presents the basic format for the element specification statement for each of the elements in Figure~\ref{fig:fig_01_02}. The first letter of the element name specifies the device and the remaining characters must assure a unique name.

\begin{figure}[h!]
	\centering
	\includegraphics[width=0.8\textwidth]{figures/fig_01_02.png}
	\label{fig:fig_01_02}
\end{figure}

\begin{table}[h!]
\centering
\label{tab:tab_01_01}
\begin{tabular}{|l|c|c|c|c|}
\hline
Element & Name & Signal Type & Control Source & Value \\ \hline
Resistor & R\textperiodcentered\textperiodcentered\textperiodcentered\ & & & $\mathrm{\Omega}$ \\ \hline
Inductor & L\textperiodcentered\textperiodcentered\textperiodcentered\ & & & H \\ \hline
Capacitor & C\textperiodcentered\textperiodcentered\textperiodcentered\ & & & F \\ \hline
Voltage source & V\textperiodcentered\textperiodcentered\textperiodcentered\ & AC or DC\textsuperscript{a} & & V\textsuperscript{b} \\ \hline
Current source & I\textperiodcentered\textperiodcentered\textperiodcentered\ & AC or DC\textsuperscript{a} & & A\textsuperscript{b} \\ \hline
VCVS & E\textperiodcentered\textperiodcentered\textperiodcentered\ & & $(cn_1,cn_2)$ & V/V \\ \hline
CCVS & H\textperiodcentered\textperiodcentered\textperiodcentered\ & & V\textperiodcentered\textperiodcentered\textperiodcentered\ & V/A \\ \hline
VCCS & G\textperiodcentered\textperiodcentered\textperiodcentered\ & & $(cn_1,cn_2)$ & A/V \\ \hline
CCCS & F\textperiodcentered\textperiodcentered\textperiodcentered\ & & V\textperiodcentered\textperiodcentered\textperiodcentered\ & A/A \\ \hline
\multicolumn{5}{|l|}{a. Time-varying signal types (SIN, PULSE, EXP, PWL, SFFM) also available.} \\
\multicolumn{5}{|l|}{b. AC signal types may specify phase angle as well as magnitude.} \\ \hline
\end{tabular}
\end{table}


\subsection{Network Theorems}
\begin{example}\label{ex:ex_01_05}
Use SPICE methods to determine the Th\'evenin equivalent circuit looking to the left through therminals 3,0 for the circuit of Figure~\ref{fig:fig_01_07}.
\begin{figure}[h!]
	\centering
	\includegraphics[width=0.8\textwidth]{figures/fig_01_07.png}
	\label{fig:fig_01_07}
\end{figure}
In SPICE independent source models, an ideal voltage source of 0V acts as a short circuit and an ideal current source of 0 A acts as an infinite impedance or open circuit. Advantage will be taken of these two features to solve the problem.\\
Load resistor $R_L$ of Figure~\ref{fig:fig_01_07}(a) is replaced by the driving point current source $I_dp$ of Figure~\ref{fig:fig_01_07}(b). The netlist code that follows forms a SPICE description of the resulting circuit. The code is set up with parameter-assigned values for $V_1$, $I_2$, and $I_{dp}$.
\begin{lstlisting}[basicstyle=\ttfamily\scriptsize\color{blue}, label={lst:ex_01_05}]
Ex1_5.CIR - Thevenin equivalent circuit
.PARAM V1value=0V I2value=0A Idpvalue=1A
V1 1 0 DC {V1value}
R1 1 2 1ohm
I2 0 2 DC {I2value}
R2 2 0 3ohm
R3 2 3 5ohm
G3 2 3 (1,0) 0.1 ; Voltage-controlled current-source
Idp 0 3 DC {Idpvalue}
.END
\end{lstlisting}
If both $V_1$ and $I_2$ are deactivated by setting V1value=I2value=0, current $I_{dp}$= 1 A must flow through the Th\'evening equivalent impedance $Z_{Th}=R_{Th}$ so that $v_3=I_{dp}R_{Th}=R_{Th}$. Execution of \texttt{<Ex1\_5.CIR>} by a SPICE program writes the values of the node voltages for nodes 1, 2, and 3 with respect to the universal ground node 0 in a file \texttt{<Ex1\_5.OUT>}. Poll the output file to find $v_3$ = V(3) = $R_{Th}$ = 5.75 $\mathrm{\Omega}$.\\
In order to determine $V_{Th}$ (open-circuit voltage between terminals 3,0), edit \verb|<Ex1_5.CIR>| to set V1value=10V, I2value=2A, and Idpvalue=0A. Execute \texttt{<Ex1\_5.CIR>} and poll the output file to find $V_{Th}=v_3$ = V(3) = 14 V.
\end{example}

\subsection{Two-Port Networks}
The \textit{z} parameters and the \textit{H} parameters can be numerically evaluated by SPICE methods. In electronics applications, the \textit{z} and \textit{h} parameters find application in analysis when small ac signals are impressed on circuits that exhibit limited-range linearity. Thus, in general, the test sources in the SPICE analysis should be of magnitudes comparable to the impressed signals of the anticipated application. Typically, the devices used in an electronic circuit will have one or more dc sources connected to bias or that place the device at a favorable point of operation. The input and output ports may be coupled by large capacitors that act to block the appearance of any dc voltages at the input and output ports while presenting negligible impedance to ac signals. Further, electronic circuits are usually frequency-sensitive so that any set of \textit{z} or \textit{h} parameters is valid for a particular frequency. Any SPICE-based evaluation of the \textit{z} and \textit{h} parameters should be capable of addressing the above outlined characteristics of electronics circuits.

\section{Semiconductor Diodes}

\subsection{Diode Terminal Characteristics}
\label{subsec:diode_term}
Use of the Fermi-Dirac probability function to predict charge neutralization give the \emph{static} (non-time-varying) equation for diode junction current:
\begin{equation}
	i_D=I_0(e^{v_D/{\eta}V_T}-1\,A
	\label{eq:eq_02_01}
\end{equation}
where
\begin{itemize}
	\item[] $V_T{\equiv}kT/q$, V
	\item[] $v_D\equiv$ diode terminal voltage, V
	\item[] $I_0\equiv$ temperature-dependent saturation current, A
	\item[] $T\equiv$ absolute temperature of \emph{p-n} junction, K
	\item[] $k\equiv$ Boltzmann's constant $(1.38{\times}10^{-23}\mathrm{J/K})$
	\item[] $q\equiv$ electron charge $(1.6{\times}10^{-19}\mathrm{C})$
	\item[] $\eta\equiv$ empirical constant, 1 for Ge and 2 for Si
\end{itemize}

\subsection{The Diode SPICE Model}
\begin{table}[h!]
	\centering
	\label{tab:tab_02_01}
	\begin{tabular}{|c|l|l|l|c|}
		\hline
		Parameter & Description & Reference & Default & Units \\ \hline
		Is & saturation current & $I_0$ of (\ref{eq:eq_02_01}) & $1{\times}10^{-14}$ & A \\ \hline
		n & emission coefficient & $\eta$ of (\ref{eq:eq_02_01}) & $1$ & \\ \hline
		BV & reverse breakdown voltage & $V_R$ of Figure~\ref{fig:fig_02_04} & $\infty$ & V \\ \hline
		IBV & reverse breakdown current & $I_R$ of Figure~\ref{fig:fig_02_04} & $1{\times}10^{-10}$ & A \\ \hline
		Rs & ohmic resistance & Subsection\ref{subsec:diode_term}) & 0 & $\mathrm{\Omega}$ \\ \hline
	\end{tabular}
\end{table}

\section{Characteristics of Bipolar Junction Transistors}

\subsection{BJT SPICE Model}
\begin{table}[h!]
	\centering
	\label{tab:tab_03_02}
	\begin{tabular}{|c|l|l|l|c|}
		\hline
		Parameter & Description & Major Impact & Default & Units \\ \hline
		Is & saturation current & $\uparrow\mathrm{Is},{\downarrow}V_\mathrm{BEQ}$ & $1{\times}10^{-16}$ & A \\ \hline
		Ikf & high current roll-off & $\downarrow\mathrm{Ikf},\downarrow\mathrm{I_C}$ & $\infty$ & A \\ \hline
		Isc & base-collector leakage & $\uparrow\mathrm{Isc},\uparrow\mathrm{I_C}$ & 0 & A \\ \hline
		Bf & forward current gain & $\uparrow\mathrm{Bf},\uparrow\mathrm{I_C}$ & 100 & \\ \hline
		Br & reverse current gain & $\uparrow\mathrm{Br},\uparrow\mathrm{rev.\,I_C}$ & 1 & \\ \hline
		Rb & base resistance & $\uparrow\mathrm{Rb},\downarrow\mathrm{di_B/dv_{BE}}$ & 0 & $\mathrm{\Omega}$ \\ \hline
		Rc & collector resistance & $\uparrow\mathrm{Rc},\uparrow\mathrm{V_{CEsat}}$ & 0 & $\mathrm{\Omega}$ \\ \hline
		Va & forward Early voltage & $\uparrow\mathrm{Va},\uparrow\mathrm{di_C/dt}$ & $\infty$ & V \\ \hline
		Cjc & base-collector capacitance & high freq. response & 0 & F \\ \hline
		Cje & base-emitter capacitance & high freq. response & 0 & F \\ \hline
	\end{tabular}
\end{table}

\end{document} 
