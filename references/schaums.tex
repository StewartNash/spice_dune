\documentclass[12pt]{article}

\usepackage{graphicx}

\title{Electronic Devices and Circuits}
\author{Jimmie J. Cathey}

\begin{document}
\maketitle

\section{Circuit Analysis}

\subsection{Spice Elements}
The passive and active circuit elements introduced in the previous section are all avaialable in SPIC modeling; however, the manner of node specification and the voltage and current sense or direction are clarified for each element by Figure~\ref{fig:fig_01_02}. THe universal ground node is assigned the number 0. Otherwise, the node numbers $n_1$ (positive node) and $n_2$ (negative node) are positive integers selected to uniquely define each node in the network. The assumed direction of positive current flow is from the node $n_1$ t node $n_2$. The four controlled sources---voltage-controlled voltage source (VCVS), current-controlled voltage source (CCVS), voltage-controlled current source (VCCS), and current-controlled current source (CCCS)--- hae the associated controlling element also shown with its nodes indicated by $cn_1$ (positive) and $cn_2$ (negative). Each element is described by an \textit{element specification statement} in the SPICE netlist code. Table~\ref{tab:tab_01_01} presents the basic format for the element specification statement for each of the elements in Figure~\ref{fig:fig_01_02}. The first letter of the element name specifies the device and the remaining characters must assure a unique name.

\begin{figure}[h!]
	\centering
	\includegraphics[width=0.8\textwidth]{figures/fig_01_02.png}
	\label{fig:fig_01_02}
\end{figure}

\begin{table}[h!]
\centering
\label{tab:tab_01_01}
\begin{tabular}{|l|c|c|c|c|}
\hline
Element & Name & Signal Type & Control Source & Value \\ \hline
Resistor & R\textperiodcentered\textperiodcentered\textperiodcentered\ & & & $\mathrm{\Omega}$ \\ \hline
\multicolumn{5}{|l|}{a. Time-varying signal types (SIN, PULSE, EXP, PWL, SFFM) also available.} \\
\multicolumn{5}{|l|}{b. AC signal types may specify phase angle as well as magnitude.} \\ \hline
\end{tabular}
\end{table}

\end{document} 
