\documentclass[12pt]{article}

\usepackage{graphicx}
\usepackage{amsthm}

\theoremstyle{definition}
\newtheorem{example}{Example}[section]

\title{Electronic Devices and Circuits}
\author{Jimmie J. Cathey}

\begin{document}
\maketitle

\section{Circuit Analysis}

\subsection{Spice Elements}
The passive and active circuit elements introduced in the previous section are all avaialable in SPIC modeling; however, the manner of node specification and the voltage and current sense or direction are clarified for each element by Figure~\ref{fig:fig_01_02}. THe universal ground node is assigned the number 0. Otherwise, the node numbers $n_1$ (positive node) and $n_2$ (negative node) are positive integers selected to uniquely define each node in the network. The assumed direction of positive current flow is from the node $n_1$ t node $n_2$. The four controlled sources---voltage-controlled voltage source (VCVS), current-controlled voltage source (CCVS), voltage-controlled current source (VCCS), and current-controlled current source (CCCS)--- hae the associated controlling element also shown with its nodes indicated by $cn_1$ (positive) and $cn_2$ (negative). Each element is described by an \textit{element specification statement} in the SPICE netlist code. Table~\ref{tab:tab_01_01} presents the basic format for the element specification statement for each of the elements in Figure~\ref{fig:fig_01_02}. The first letter of the element name specifies the device and the remaining characters must assure a unique name.

\begin{figure}[h!]
	\centering
	\includegraphics[width=0.8\textwidth]{figures/fig_01_02.png}
	\label{fig:fig_01_02}
\end{figure}

\begin{table}[h!]
\centering
\label{tab:tab_01_01}
\begin{tabular}{|l|c|c|c|c|}
\hline
Element & Name & Signal Type & Control Source & Value \\ \hline
Resistor & R\textperiodcentered\textperiodcentered\textperiodcentered\ & & & $\mathrm{\Omega}$ \\ \hline
Inductor & L\textperiodcentered\textperiodcentered\textperiodcentered\ & & & H \\ \hline
Capacitor & C\textperiodcentered\textperiodcentered\textperiodcentered\ & & & F \\ \hline
Voltage source & V\textperiodcentered\textperiodcentered\textperiodcentered\ & AC or DC\textsuperscript{a} & & V\textsuperscript{b} \\ \hline
Current source & I\textperiodcentered\textperiodcentered\textperiodcentered\ & AC or DC\textsuperscript{a} & & A\textsuperscript{b} \\ \hline
VCVS & E\textperiodcentered\textperiodcentered\textperiodcentered\ & & $(cn_1,cn_2)$ & V/V \\ \hline
CCVS & H\textperiodcentered\textperiodcentered\textperiodcentered\ & & V\textperiodcentered\textperiodcentered\textperiodcentered\ & V/A \\ \hline
VCCS & G\textperiodcentered\textperiodcentered\textperiodcentered\ & & $(cn_1,cn_2)$ & A/V \\ \hline
CCCS & F\textperiodcentered\textperiodcentered\textperiodcentered\ & & V\textperiodcentered\textperiodcentered\textperiodcentered\ & A/A \\ \hline
\multicolumn{5}{|l|}{a. Time-varying signal types (SIN, PULSE, EXP, PWL, SFFM) also available.} \\
\multicolumn{5}{|l|}{b. AC signal types may specify phase angle as well as magnitude.} \\ \hline
\end{tabular}
\end{table}

\begin{example}\label{ex:ex_01_05}
Use SPICE methods to determine the Th\'evenin equivalent circuit looking to the left through therminals 3,0 for the circuit of Figure~\ref{fig:fig_01_07}.
\begin{figure}[h!]
	\centering
	\includegraphics[width=0.8\textwidth]{figures/fig_01_07.png}
	\label{fig:fig_01_07}
\end{figure}
In SPICE independent source models, an ideal voltage source of 0V acts as a short circuit and an ideal current source of 0 A acts as an infinite impedance or open circuit. Advantage will be taken of these two features to solve the problem.\\
Load resistor $R_L$ of Figure~\ref{fig:fig_01_07}(a) is replaced by the driving point current source $I_dp$ of Figure~\ref{fig:fig_01_07}(b). The netlist code that follows forms a SPICE description of the resulting circuit. The code is set up with parameter-assigned values for $V_1$, $I_2$, and $I_{dp}$.\\
If both $V_1$ and $I_2$ are deactivated by setting V1value=I2value=0, current $I_{dp}$= 1 A must flow through the Th\'evening equivalent impedance $Z_{Th}=R_{Th}$ so that $v_3=I_{dp}R_{Th}=R_{Th}$.
\end{example}

\end{document} 
