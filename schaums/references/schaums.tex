\documentclass[12pt]{article}

\usepackage{graphicx}
\usepackage{amsthm}
\usepackage{listings}
\usepackage{xcolor}

\theoremstyle{definition}
\newtheorem{example}{Example}[section]

% Custom command for editor
\newcommand{\editor}[1]{\gdef\@editor{#1}}
\newcommand{\printeditor}{%
  \par\vspace{1em}
  \textbf{Editor:} \@editor
}

\title{Electronic Devices and Circuits}
\author{Jimmie J. Cathey}

\editor{Stewart Nash}

\begin{document}
\maketitle
\printeditor

% Chapter 1
\section{Circuit Analysis}

\subsection{Spice Elements}
The passive and active circuit elements introduced in the previous section are all avaialable in SPIC modeling; however, the manner of node specification and the voltage and current sense or direction are clarified for each element by Figure~\ref{fig:fig_01_02}. THe universal ground node is assigned the number 0. Otherwise, the node numbers $n_1$ (positive node) and $n_2$ (negative node) are positive integers selected to uniquely define each node in the network. The assumed direction of positive current flow is from the node $n_1$ t node $n_2$. The four controlled sources---voltage-controlled voltage source (VCVS), current-controlled voltage source (CCVS), voltage-controlled current source (VCCS), and current-controlled current source (CCCS)--- hae the associated controlling element also shown with its nodes indicated by $cn_1$ (positive) and $cn_2$ (negative). Each element is described by an \textit{element specification statement} in the SPICE netlist code. Table~\ref{tab:tab_01_01} presents the basic format for the element specification statement for each of the elements in Figure~\ref{fig:fig_01_02}. The first letter of the element name specifies the device and the remaining characters must assure a unique name.

\begin{figure}[h!]
	\centering
	\includegraphics[width=0.8\textwidth]{figures/fig_01_02.png}
	\caption{}
	\label{fig:fig_01_02}
\end{figure}

\begin{table}[h!]
\centering
\caption{}
\label{tab:tab_01_01}
\begin{tabular}{|l|c|c|c|c|}
\hline
Element & Name & Signal Type & Control Source & Value \\ \hline
Resistor & R\textperiodcentered\textperiodcentered\textperiodcentered\ & & & $\mathrm{\Omega}$ \\ \hline
Inductor & L\textperiodcentered\textperiodcentered\textperiodcentered\ & & & H \\ \hline
Capacitor & C\textperiodcentered\textperiodcentered\textperiodcentered\ & & & F \\ \hline
Voltage source & V\textperiodcentered\textperiodcentered\textperiodcentered\ & AC or DC\textsuperscript{a} & & V\textsuperscript{b} \\ \hline
Current source & I\textperiodcentered\textperiodcentered\textperiodcentered\ & AC or DC\textsuperscript{a} & & A\textsuperscript{b} \\ \hline
VCVS & E\textperiodcentered\textperiodcentered\textperiodcentered\ & & $(cn_1,cn_2)$ & V/V \\ \hline
CCVS & H\textperiodcentered\textperiodcentered\textperiodcentered\ & & V\textperiodcentered\textperiodcentered\textperiodcentered\ & V/A \\ \hline
VCCS & G\textperiodcentered\textperiodcentered\textperiodcentered\ & & $(cn_1,cn_2)$ & A/V \\ \hline
CCCS & F\textperiodcentered\textperiodcentered\textperiodcentered\ & & V\textperiodcentered\textperiodcentered\textperiodcentered\ & A/A \\ \hline
\multicolumn{5}{|l|}{a. Time-varying signal types (SIN, PULSE, EXP, PWL, SFFM) also available.} \\
\multicolumn{5}{|l|}{b. AC signal types may specify phase angle as well as magnitude.} \\ \hline
\end{tabular}
\end{table}


\subsection{Network Theorems}
\begin{example}\label{ex:ex_01_05}
Use SPICE methods to determine the Th\'evenin equivalent circuit looking to the left through therminals 3,0 for the circuit of Figure~\ref{fig:fig_01_07}.
\begin{figure}[h!]
	\centering
	\includegraphics[width=0.8\textwidth]{figures/fig_01_07.png}
	\caption{}
	\label{fig:fig_01_07}
\end{figure}
In SPICE independent source models, an ideal voltage source of 0V acts as a short circuit and an ideal current source of 0 A acts as an infinite impedance or open circuit. Advantage will be taken of these two features to solve the problem.\\
Load resistor $R_L$ of Figure~\ref{fig:fig_01_07}(a) is replaced by the driving point current source $I_dp$ of Figure~\ref{fig:fig_01_07}(b). The netlist code that follows forms a SPICE description of the resulting circuit. The code is set up with parameter-assigned values for $V_1$, $I_2$, and $I_{dp}$.
\begin{lstlisting}[basicstyle=\ttfamily\scriptsize\color{blue}, label={lst:ex_01_05}]
Ex1_5.CIR - Thevenin equivalent circuit
.PARAM V1value=0V I2value=0A Idpvalue=1A
V1 1 0 DC {V1value}
R1 1 2 1ohm
I2 0 2 DC {I2value}
R2 2 0 3ohm
R3 2 3 5ohm
G3 2 3 (1,0) 0.1 ; Voltage-controlled current-source
Idp 0 3 DC {Idpvalue}
.END
\end{lstlisting}
If both $V_1$ and $I_2$ are deactivated by setting V1value=I2value=0, current $I_{dp}$= 1 A must flow through the Th\'evening equivalent impedance $Z_{Th}=R_{Th}$ so that $v_3=I_{dp}R_{Th}=R_{Th}$. Execution of \texttt{<Ex1\_5.CIR>} by a SPICE program writes the values of the node voltages for nodes 1, 2, and 3 with respect to the universal ground node 0 in a file \texttt{<Ex1\_5.OUT>}. Poll the output file to find $v_3$ = V(3) = $R_{Th}$ = 5.75 $\mathrm{\Omega}$.\\
In order to determine $V_{Th}$ (open-circuit voltage between terminals 3,0), edit \verb|<Ex1_5.CIR>| to set V1value=10V, I2value=2A, and Idpvalue=0A. Execute \texttt{<Ex1\_5.CIR>} and poll the output file to find $V_{Th}=v_3$ = V(3) = 14 V.
\end{example}

\subsection{Two-Port Networks}
The \textit{z} parameters and the \textit{H} parameters can be numerically evaluated by SPICE methods. In electronics applications, the \textit{z} and \textit{h} parameters find application in analysis when small ac signals are impressed on circuits that exhibit limited-range linearity. Thus, in general, the test sources in the SPICE analysis should be of magnitudes comparable to the impressed signals of the anticipated application. Typically, the devices used in an electronic circuit will have one or more dc sources connected to bias or that place the device at a favorable point of operation. The input and output ports may be coupled by large capacitors that act to block the appearance of any dc voltages at the input and output ports while presenting negligible impedance to ac signals. Further, electronic circuits are usually frequency-sensitive so that any set of \textit{z} or \textit{h} parameters is valid for a particular frequency. Any SPICE-based evaluation of the \textit{z} and \textit{h} parameters should be capable of addressing the above outlined characteristics of electronics circuits.

% Chapter 2
\section{Semiconductor Diodes}

\subsection{The Ideal Diode}
\begin{figure}[h!]
	\centering
	\includegraphics[width=0.8\textwidth]{figures/fig_02_01.png}
	\caption{}
	\label{fig:fig_02_01}
\end{figure}

\subsection{Diode Terminal Characteristics}
\label{subsec:diode_term}
Use of the Fermi-Dirac probability function to predict charge neutralization give the \emph{static} (non-time-varying) equation for diode junction current:
\begin{equation}
	i_D=I_0(e^{v_D/{\eta}V_T}-1\,A
	\label{eq:eq_02_01}
\end{equation}
where
\begin{itemize}
	\item[] $V_T{\equiv}kT/q$, V
	\item[] $v_D\equiv$ diode terminal voltage, V
	\item[] $I_0\equiv$ temperature-dependent saturation current, A
	\item[] $T\equiv$ absolute temperature of \emph{p-n} junction, K
	\item[] $k\equiv$ Boltzmann's constant $(1.38{\times}10^{-23}\mathrm{J/K})$
	\item[] $q\equiv$ electron charge $(1.6{\times}10^{-19}\mathrm{C})$
	\item[] $\eta\equiv$ empirical constant, 1 for Ge and 2 for Si
\end{itemize}

\subsection{The Diode SPICE Model}
The element specification statement for a diode must explicitly name a model even if the default model parameters are intended for use. The general form of the diode specification statement is a as follows, where the \emph{model name} is arbitrarily chosen:
\begin{center}
$\mathrm{D}{\cdots}n_1n_2$ \emph{model name}
\end{center}
Node $n_1$ is the anode and node $n_2$ is the cathode of the diode. Positive current and voltage directions are clarified by Figure~\ref{fig:fig_02_01}(b).\\
In addition, the .MODEL control statement must be added to the netlist code even if the default parameters are acceptable. This control statement is
\begin{center}
MODEL \emph{model name} D \emph{(parameters)}
\end{center}
If the parameters field is left blank, default values are assigned. Otherwise, the parameters field contains the number of desired specifications in the format \emph{parameter name = value}. Specific parameters of concern in this book are documented by Table~\ref{tab:tab_02_01}.
\begin{table}[h!]
	\centering
	\caption{}
	\label{tab:tab_02_01}
	\begin{tabular}{|c|l|l|l|c|}
		\hline
		Parameter & Description & Reference & Default & Units \\ \hline
		Is & saturation current & $I_0$ of (\ref{eq:eq_02_01}) & $1{\times}10^{-14}$ & A \\ \hline
		n & emission coefficient & $\eta$ of (\ref{eq:eq_02_01}) & $1$ & \\ \hline
		BV & reverse breakdown voltage & $V_R$ of Figure~\ref{fig:fig_02_04} & $\infty$ & V \\ \hline
		IBV & reverse breakdown current & $I_R$ of Figure~\ref{fig:fig_02_04} & $1{\times}10^{-10}$ & A \\ \hline
		Rs & ohmic resistance & Subsection\ref{subsec:diode_term}) & 0 & $\mathrm{\Omega}$ \\ \hline
	\end{tabular}
\end{table}

% Chapter 3
\section{Characteristics of Bipolar Junction Transistors}

\subsection{BJT SPICE Model}
The element specification statement for a BJT must explicitly name a model even if the default model parameters are intended for use. The general form of the transistor specification statement is as follows:
\begin{center}
$\mathrm{Q}{\cdots}n_1n_2n_3$ \emph{model name}
\end{center}
Nodes $n_1$, $n_2$, and $n_3$ belong to the collector, base, and emitter, respectively. The \emph{model name} is an arbitrary selection of alpha and numeric characters to uniquely identify the model. Positive current and voltage directions for the \textit{pnp} and \textit{npn} transistor are clarified by Figure~\ref{fig:fig_03_04}.\\
In addition, a .MODEL control statement must be added to the netlist code. This control statement specifies whether the transistor is \emph{pnp} or \emph{npn} and thus has one of the following two forms:
\begin{center}
MODEL \emph{model name} PNP \emph{(parameters)}\\
MODEL \emph{model name} NPN \emph{(parameters)}
\end{center}
If the parameter field is left blank, default values are assigned. Non-default desired parameter specifications are entered in the parameter field using the format \emph{parameter name = value}. Specific parameters that are of concern in this book are documented in Table~\ref{tab:tab_03_02}.\\
All parameter values are entered with positive values regardless of whether the transistor is \textit{pnp} or \textit{npn}. Two transistor models will be used in this chapter--\emph{generic model} and \emph{default model}--as introduced in Example~\ref{ex:ex_03_02}.
\begin{table}[h!]
	\centering
	\caption{}
	\label{tab:tab_03_02}
	\begin{tabular}{|c|l|l|l|c|}
		\hline
		Parameter & Description & Major Impact & Default & Units \\ \hline
		Is & saturation current & $\uparrow\mathrm{Is},{\downarrow}V_\mathrm{BEQ}$ & $1{\times}10^{-16}$ & A \\ \hline
		Ikf & high current roll-off & $\downarrow\mathrm{Ikf},\downarrow\mathrm{I_C}$ & $\infty$ & A \\ \hline
		Isc & base-collector leakage & $\uparrow\mathrm{Isc},\uparrow\mathrm{I_C}$ & 0 & A \\ \hline
		Bf & forward current gain & $\uparrow\mathrm{Bf},\uparrow\mathrm{I_C}$ & 100 & \\ \hline
		Br & reverse current gain & $\uparrow\mathrm{Br},\uparrow\mathrm{rev.\,I_C}$ & 1 & \\ \hline
		Rb & base resistance & $\uparrow\mathrm{Rb},\downarrow\mathrm{di_B/dv_{BE}}$ & 0 & $\mathrm{\Omega}$ \\ \hline
		Rc & collector resistance & $\uparrow\mathrm{Rc},\uparrow\mathrm{V_{CEsat}}$ & 0 & $\mathrm{\Omega}$ \\ \hline
		Va & forward Early voltage & $\uparrow\mathrm{Va},\uparrow\mathrm{di_C/dt}$ & $\infty$ & V \\ \hline
		Cjc & base-collector capacitance & high freq. response & 0 & F \\ \hline
		Cje & base-emitter capacitance & high freq. response & 0 & F \\ \hline
	\end{tabular}
\end{table}
\begin{example}\label{ex:ex_03_02}
Use SPICE methods to generate the CE collector characteristics for an \textit{npn} transistor characterized by (a) the default parameter values and (b) a reasonable set of values for the parameters appearing in Table~\ref{tab:tab_03_02}.
\begin{figure}[h!]
	\centering
	\includegraphics[width=0.8\textwidth]{figures/fig_03_05_a.png}
	\caption{}
	\label{fig:fig_03_05_a}
\end{figure}
\begin{lstlisting}[basicstyle=\ttfamily\scriptsize\color{blue}, label={lst:ex_03_02}]
Ex3_2.CIR
Ib 0 1 0uA
Q 2 1 0 QNPN
*Q 2 1 0 QNPNG
VC 2 0 0V
.MODEL QNPN NPN() ; Default BJT
*.MODEL QNPNG NPN(Is=10fA Ikf=150mA Isc=10fA Bf=150
*+ Br=3 Rb=1ohm Rc=1ohm Va=30V Cjc=10pF Cje=15pf)
.DC VC 0V 15V 1V Ib 0uA 150uA 25uA
.PROBE
.END
\end{lstlisting}
\end{example}

% Chapter 4
\section{Characteristics of Field-Effect Transistors and Triodes}

\subsection{JFET SPICE Model}
The element specification statement for a JFET must explicitly assign a \emph{model name} that is an arbitrary selection of alpha and numeric characters. The general form is
\begin{center}
$\mathrm{J}{\cdots}n_1n_2n_3$ \emph{model name}
\end{center}
Nodes $n_1$, $n_2$, and $n_3$ belong to the drain, gate, and source, respectively. Only the \emph{n}-channel JFET is addressed in this book. Positive voltage and current directions for the device are clarified by Figure~\ref{fig:fig_04_03}.
\begin{figure}[h!]
	\centering
	\includegraphics[width=0.8\textwidth]{figures/fig_04_03.png}
	\caption{}
	\label{fig:fig_04_03}
\end{figure}
A .MODEL control statement must appear in the netlist code for a JFET circuit. The control statement has the following format:
\begin{center}
MODEL \emph{model name} NJF \emph{(parameters)}
\end{center}
If the parameter field is left blank, default values are assigned. Nondefault parameters are entered in the parameter field using the format \emph{parameter name = value}. The specific parameters of concern in the book are documented by Table~\ref{tab:tab_04_02}. The SPICE model describes the JFET in the pinchoff region by
\begin{equation}
	i_d=\frac{I_{DSS}}{(V_{to})^2}(V_{to}+v_{GS})^2=Beta(V_{to}+v_{GS})^2
\end{equation}
\begin{table}[h!]
	\centering
	\caption{}
	\label{tab:tab_04_02}
	\begin{tabular}{|c|l|l|l|l|}
		\hline
		Parameter & Description & Major Impact & Default & Units \\ \hline
		Vto & pinchoff voltage & shorted-gate current & -2 & V \\ \hline
		Beta & transcond. coeff. & shorted-gate current & 0.0001 & $\mathrm{A/V^2}$ \\ \hline
		Rd & drain resistance & current limit & 0 & $\mathrm{\Omega}$ \\ \hline
		Rs & source resistance & current limit & 0 & $\mathrm{\Omega}$ \\ \hline
		CGS & gate-source cap. & high frequency & 0 & F \\ \hline
		CGD & gate-drain cap. & high frequency & 0 & F \\ \hline
	\end{tabular}
\end{table}
\begin{example}\label{ex:ex_04_01}
Use SPICE methods to generate (a) the CS drain characteristics and (b) the transfer characteristic for an \textit{n}-channel JFET that has the parameter values Vto= -4V, Beta = 0.0005 A/V\textsuperscript{2}, Rd = 1$\mathrm{\Omega}$, Rs = 1$\mathrm{\Omega}$, and CGS = CGD = 2pF.\\
(a) Figure~\ref{fig:fig_04_04}(a) shows a connection method for measurement of both the drain characteristics and the transfer characteristic. The following netlist code generates the drain characteristics that have been plotted using the Probe feature of PSpice as Figure~\ref{fig:fig_04_04}(b).
\begin{lstlisting}[basicstyle=\ttfamily\scriptsize\color{blue}, label={lst:ex_04_01_a}]
Ex4_1a.CIR - JFET drain characteristics
vGS 1 0 0V
vDS 2 0 0V
J 2 1 0 NJFET
.MODEL NJFET NJF ( Vto=-4V Beta=0.0005ApVsq
+ Rd=1ohm Rs=1ohm CGS=2pF CGD=2pF)
.DC vDS 0V 25V 0.5V vGS 0V -4V 0.5V
.PROBE
.END
\end{lstlisting}
\begin{figure}[h!]
	\centering
	\includegraphics[width=0.8\textwidth]{figures/fig_04_04.png}
	\label{fig:fig_04_04}
\end{figure}
\end{example}

\subsection{MOSFET SPICE Model}
The element specification statement for a MOSFET must explicitly assign a \emph{model name} (an arbitrary selection of alpha and numeric characters) having the general form
\begin{center}
$\mathrm{M}{\cdots}n_1n_2n_3n_4$ \emph{model name}
\end{center}
Nodes $n_1$, $n_2$, $n_3$, and $n_4$ belong to the drain, gate, source, and substrate, respectively. Only the \emph{n}-channel MOSFET is addressed where the device positive voltage and current directions are clarified by Figure~\ref{fig:fig_04_10}.
\begin{figure}[h!]
	\centering
	\includegraphics[width=0.8\textwidth]{figures/fig_04_10.png}
	\caption{}
	\label{fig:fig_04_10}
\end{figure}
Format of the .MODEL control statement that must appear in the netlist code for a MOSFET circuit is as follows:
\begin{center}
MODEL \emph{model name} NMOS \emph{(parameters)}
\end{center}
A blank parameter field results in assignment of default paramter values. Nondefault parameters are entered in the parameter field as \emph{parameter name = value}. The specific parameters of concern in the book are documented by Table~\ref{tab:tab_04_03}. The SPICE model characterizes the enhancement mode MOSFET in the pinchoff region by
\begin{equation}
	i_d=\frac{I_{D\mathrm{on}}}{(V_T^2}(v_{GS}-V_T)^2=\frac{Kp}{2}(v_{GS}-V_T)^2
\end{equation}
\begin{table}[h!]
	\centering
	\caption{}
	\label{tab:tab_04_03}
	\begin{tabular}{|c|l|l|l|l|}
		\hline
		Parameter & Description & Default & Units \\ \hline
		Vto & Threshold voltage & 0 & V \\ \hline
		Kp & Transcond. coeff. & $2{\times}10^{-5}$ & $\mathrm{A/V^2}$ \\ \hline
		Rd & Drain resistance & 0 & $\mathrm{\Omega}$ \\ \hline
		Rg & Gate resistance & 0 & $\mathrm{\Omega}$ \\ \hline
	\end{tabular}
\end{table}

% Chapter 5
\section{Transistor Bias Considerations}

\subsection{Stability-Factor Analysis}
\begin{equation}
	S_I=\left.\frac{{\Delta}I_{CQ}}{{\Delta}I_{CBO}}\right|_Q\approx\left.\frac{{\partial}I_{CQ}}{{\partial}I_{CBO}}\right|_Q
	\label{eq:eq_05_11}
\end{equation}


\subsection{Parameter Variation Analysis with SPICE}
PSpice offers two features that allow direct study of circuit performance change due to parameters variation. The first of these features is simply called \emph{sensitivity analysis}. It is invoked by a control statement of the following format:
\begin{center}
.SENS \textit{sensitive variable}
\end{center}
The \textit{sensitive variable} can be any node voltage or the current through any independent voltage source. A table is generated in the output file that gives the sensitivity of the sensitive variable to each parameter (specified or default) in the model of all BJTs and diodes that are directly comparable with (\ref{eq:eq_05_11}) and

% Chapter 6
\section{Small-Signal Midfrequency BJT Amplifiers}

\subsection{BJT Amplifier Analysis with SPICE}
Since SPICE models of the BJT (see Chapter 3) provide the device terminal characteristics, a transistor amplifier can be properly biased and a time-varying input signal can be directly applied to the completely modeled amplifier circuit. Any desired signal that results can be measured directly in the time domain to form signal ratios that yield the current and voltage gains. With such modeling, any signal distortion that results from nonlinear operation of the BJT is readily apparent from inspection of the signal-time plots. Such an analysis approach is the analytical equivalent of laboratory operation of the amplifier where the time plot of signals is analogous to oscilloscope observation of the amplifier circuit signals.\\

% Chapter 7
\section{Small-Signal Midfrequency FET and Triode Amplifiers}

\subsection{FET Amplifier Gain Calculation with SPICE}
SPICE models of the JFET and MOSFET (introduced in Chapter 4) provide the terminal characteristics of the devices; thus, an amplifier can be properly biased and time-varying input signal directly applied to the completely modeled amplifier circuit. Such a simulation is the analytical equivalent of laboratory amplifier circuit operation. Any desired signal can be measured directly in the time domain to form signal ratios that yield current and voltage gains. Any signal distoration that may result from device nonlinearity is readily apparent from inspection of the signal time plots. 

% Chapter 8
\section{Frequency Effects in Amplifiers}

\subsection{Frequency Response Using SPICE}
SPICE methods offer a frequency sweep option that allows a small-signal, sinusoidal steady-state analysis of a circuit. The frequency sweep is invoked by a control statement of the following format:
\begin{center}
.AC DEC \emph{points start freq end freq}
\end{center}
Node voltages and device current are inherently complex number values. The magnitudes and phase angles of calculated quantities can be retrieved by the Probe feature of PSpice by appending a $p$ and $n$, respectively, to the variable. For example, magnitude and phase angle of the voltage between nodes 2 and 3 are specified by Vm(2,3) and Vp(2,3).
\begin{example}
	For the BJT amplifier circuit of Fig. 3-10, assume $C_c\rightarrow\infty$. The small-signal equivalent circuit is given by Fig. 8-4 where $R_B=R_1{\parallel}R_2$. Let $h_{oe}=h_{re}=0$, $h_{fe}=90$, $R_1=1\mathrm{k\Omega}$, $R_2=16\mathrm{k\Omega}$, $R_E=500\mathrm{\Omega}$, $C_E=330\mathrm{{\mu}F}$, $R_C=1\mathrm{k\Omega}$, and $R_L=10\mathrm{k\Omega}$. Use SPICE methods to determine the low-frequency cutoff point.\\
	The netlist code that follows describes the circuit
	\begin{lstlisting}[basicstyle=\ttfamily\scriptsize\color{blue}, label={lst:ex_08_09}]
Ex8_9.CIR
vi 1 0 AC 0.250V
R1 1 0 1kohm
R2 1 0 16kohm
Vsen 1 2 DC 0V
Rhie 2 3 200ohm
Fhfe 3 4 Vsen 90
RE 3 0 500ohm
CE 3 0 330uF
RC 4 0 1kohm
RL 4 0 10kohm
.AC DEC 25 10Hz 10kHz
.PROBE
.END
	\end{lstlisting}
	Execute <Ex9\_9.CIR> and use the Probe feature of PSPice to yield the plots of Fig. 8-13. Fromt he marked points, it is seen that the low-frequency cutoff is $f_L=214.4\mathrm{Hz}$, where the voltage gain has a value of $A_{vL}=289.7$.
\end{example}

The above example utilized the small-signal equivalent circuit. Small-signal analysis frequency sensitivity can also be implemented using the SPICE model of the transistor directly.

% Chapter 9
\section{Operational Amplifiers}

\subsection{SPICE Op Amp Model}
Figure 9-1(a) presents the equivalent circuit model of the op amp using a VCVS to implement the gain. This circuit is easily realized by SPICE methods using the VCVS model of Fig. 1-2 and Table 1-1. It is frequently convenient to describe the op am through the use of a subcircuit as illustrated by the following netlist code:
\begin{verbatim}
.SUBCKT OPAMP 1   2    3   4
*       Model Inv NInv Out Com
Rd 1 2 500kohm
E  5 4 (1,2) -1e5
Ro 5 3 100ohm
.ENDS OPAMP
\end{verbatim}
The nodes are labeled in Fig. 9-1(a). Input impedance ($R_d$ = 500 k$\Omega$), output impedance ($R_o$ = 100 $\Omega$), and open-loop voltage gain ($A_{OL}=-1\times10^5$) are typical values that can be changed if an application warrants. Also, SPICE libraries usually contain subcircuit models of commercially available op amps that can be utilized.
\begin{example}\label{ex:ex_09_11}
	Use SPICE methods to model the noninverting amplifier of Fig. 9-3 if the op amp has the parameter values of the subcircuit OPAMP above. Let $v_s=0.5\sin{(200{\pi}t)}$ V, $R_1=1\,\mathrm{k\Omega}$, and $R_2=10\,\mathrm{k\Omega}$. Verify that hte voltage gain predicted by (9.7) results.\\
Netlist code describing the circuit is shown below:
\begin{lstlisting}[basicstyle=\ttfamily\scriptsize\color{blue}, label={lst:ex_09_11}]
Ex9_11.CIR
vs 1 0 SIN(0V 0.5V 1kHz)
R1 2 0 1kohm
R2 3 2 10kohm
XA 1 2 3 0 OPAMP
.SUBCKT OPAMP 1   2    3   4
*       Model Inv NInv Out Com
Rd 1 2 500kohm
E 5 4 (1,2) -1e5
Ro 5 3 100ohm
.ENDS OPAMP
.TRAN 1us 2ms
.PROBE
.END
\end{lstlisting}
Execute <Ex9\_11.CIR> and use the Probe feature of PSpice to plot Fig. 9-10. By use of the marked values Fig. 9-10,
\begin{equation}
	A_v=\frac{5.5}{0.5}=11
\end{equation}
The voltage gain predicted by (9.7) is
\begin{equation}
	A_v=1+\frac{R_2}{R_1}=1+\frac{10\times10^3}{1\times10^3}=11
\end{equation}
Hence, (9.7) is validated.
\end{example}

% Chapter 10
\section{Switched Mode Power Supplies}

\subsection{Buck Converter}
The SMPS circuit of Figure~\ref{fig:fig_10_02}, known as a \emph{buck converter}, produces an average value output voltage $V_2=\langle{v_2}\rangle\leq{V_1}$.
\begin{figure}[h!]
	\centering
	\includegraphics[width=0.8\textwidth]{figures/fig_10_02.png}
	\caption{Buck converter}
	\label{fig:fig_10_02}
\end{figure}
%\begin{figure}[h!]
%	\centering
%	\includegraphics[width=0.8\textwidth]{figures/fig_10_03.png}
%	\caption{Buck converter waveform}
%	\label{fig:fig_10_03}
%\end{figure}
%TODO: Include Figure 10-03.

\subsection{Boost Converter}
The \emph{boost converter} SMPS circuit of Figure~\ref{fig:fig_10_04} produces an average value output voltage $V_2=\langle{v_2}\rangle>{V_1}$.
\begin{figure}[h!]
	\centering
	\includegraphics[width=0.8\textwidth]{figures/fig_10_04.png}
	\caption{Boost converter}
	\label{fig:fig_10_04}
\end{figure}
%\begin{figure}[h!]
%	\centering
%	\includegraphics[width=0.8\textwidth]{figures/fig_10_05.png}
%	\caption{Boost converter waveform}
%	\label{fig:fig_10_05}
%\end{figure}
%TODO: Include Figure 10-05.

\subsection{SPICE Analysis of SMPS}
For simulation of near ideal (lossless) SMPS, the switch element $Q$ can readily be modeled using the PSpice voltage-controlled switch. The element specification statement for the voltage-controlled switch has the form
\begin{center}
$\mathrm{S}{\cdots}n_1\,n_2\,c_1\,c_2$ VCS
\end{center}
Any alpha-numeric combination suffix can follow S to uniquely specify the voltage-controlled switch. The nodes are clarified by Figure~\ref{fig:fig_10_08}. A fast rise and fall time (5 ns), 1-V pulse should be used for the control voltage $v_{SW}$. Accepting the default ON state and OFF state control voltages of 1 V and 0 V, respectively, results in duty cycle ON time approximately equal to the pulse duration. For minimum conduction losses, the ON state resistance of the voltage control switch should be specified in the .MODEL statement by
\begin{center}
.MODEL VCS VSWITCH (RON = 1e-6)
\end{center}
%\begin{figure}[h!]
%	\centering
%	\includegraphics[width=0.8\textwidth]{figures/fig_10_08.png}
%	\label{fig:fig_10_08}
%\end{figure}
%TODO: Include Figure 10-08.
\begin{example}\label{ex:ex_10_05}
Use SPICE methods to model the buck converter of Figure~\ref{fig:fig_10_02}; let $D=0.5$, $f_s=25$ kHz, $L=100\,\mu\mathrm{H}$, $C=50\,\mu\mathrm{F}$, and $R_L=5\,\Omega$. Generate the set of waveforms analogous to Figure~\ref{fig:fig_10_03}.\\
The netlist code follows, where the initial conditions on inductor current and capacitor voltage were determined after running a large integer number of cycles to find the repetitive values.
\begin{lstlisting}[basicstyle=\ttfamily\scriptsize\color{blue}, label={lst:ex_10_05}]
Ex10_5.CIR
* BUCK CONVERTER
* D=DUTY CYCLE, fs=SWITCHING FREQUENCY
.PARAM D=-.5 fs=25e3Hz
V1 1 0 DC 12V
SW 1 2 4 2 VCS
VSW 4 2 PULSE(0V 1V 0s 5ns 5ns {D/fs} {1/fs})
L 2 3 100uH IC=0.6A
D 0 2 DMOD
C 3 0 50uF IC=6V
RL 3 0 5ohm
.MODEL DMOD D(N=0.01)
.MODEL VCS VSWITCH (RON=1e-6ohm)
.TRAN 5us 0.2ms 0s 100ns UIC
.PROBE
.END
\end{lstlisting}
Execute Listing~\ref{lst:ex_10_05} \verb|<Ex10_5.CIR>| and use the Probe feature of PSpice to plot the waveforms of Figure~\ref{fig:fig_10_09}.
\begin{figure}[h!]
	\centering
	\includegraphics[width=0.8\textwidth]{figures/fig_10_09.png}
	\label{fig:fig_10_09}
\end{figure}
\end{example}

\end{document} 
